% En este archivo definimos la verdadera salsa!
\section{Aspectos técnicos}

Asumiendo que se ha hablado de Coq, es necesario hablar de Mtac2, y mencionar
las diferentes estructuras que utilizamos (\mintinline{Coq}{MTele_in}, etc).

Con todo eso en la bolsa, el verdadero desafio es el de explicar la forma en que
nos aproximamos a esta solución.


Me gustaría hablar de como decidimos la heurística que estabamos utilizando,
onda, cómo llegamos a que eso tenía sentido y en que casos se aplicaba.
En el mejor de los casos estaría bueno poder idear un estilo de \textit{syntax
  sugar} para los telescopios.
Tal vez sea una buena idea volver a las bases de los telescopios y repensar esto
en papel para ver como lo pasamos una fórmula.

Se puede mencionar que tuve que leer código de Mtac2 y tengo unos pull request
\textbf{mínimos}.

\subsection{Bind}

% TODO Definir todas las cosas de Telescopios! telescope.v
% Se asume que se han explicado todas las cosas de MTele.v

Para comenzar a estudiar el problema es mejor centrarse en casos más simples que
podamos razonar. La primera función interesante que podemos liftear es
\mintinline{Coq}{bind}.
\begin{center}
  \begin{minted}{Coq}
    forall A B : Type, M A -> (A -> M B) -> M B
  \end{minted}
\end{center}
%$$ \forall A\, B : \text{Type}, \, M A \rightarrow (A \rightarrow M B)
%\rightarrow M B $$
Ahora es el momento de decidir exactamente qué buscamos modificar de una
función. El tipo que nos interesa tener para nuestro nuevo
\mintinline{coq}{bind}, al cual llamaremos \mintinline{coq}{mbind}, es
\begin{center}
  \begin{minted}{Coq}
    mbind : forall m : MTele, A B : MTele_ty, MFA A -> (A -> MFA B) -> MFA B
  \end{minted}
\end{center}
$$\text{mbind} : \forall \, m : \text{MTele} , \, A\, B : \text{MTele\_ty}
$$