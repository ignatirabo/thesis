%*******************************************************
% Abstract
%*******************************************************
%\renewcommand{\abstractname}{Abstract}
\pdfbookmark[1]{Resumen}{Resumen}
% \addcontentsline{toc}{chapter}{\tocEntry{Abstract}}
\begingroup
\let\clearpage\relax
\let\cleardoublepage\relax
\let\cleardoublepage\relax

% I try to have four sentences in my abstract.
% The first states the problem.
% The second states why the problem is a problem.
% The third is my startling sentence.
% The fourth states the implication of my startling sentence.


\chapter*{Resumen}

Los meta-lenguajes de programación son una parte esencial de los asistentes de pruebas. Estos meta-lenguajes deben lidiar con multiples problemas y para esto utilizan diferentes mecanismos. En particular, el meta-lenguaje \Mtac en Coq utiliza mónadas como una forma de añadir tipado descriptivo a las tácticas de Coq.

Estos programas monádicos no son fáciles de desarrollar.
En parte, vincular cómputos monádicos puede ser una tarea complicada con el sistema de tipos de Coq, donde se utilizan tipos dependientes.
Solucionar estos problemas requiere de esfuerzo programacional inútil, no relacionado al objetivo.

Utilizando \mtac podemos crear un nuevo metaprograma que generalice la signatura de las funciones de forma casi automática, permitiendo una comunicación fluida entre declaraciones monádicas. % seamless interaction/communication

Con este trabajo, el programador puede concentrarse en lo verdaderamente importante y olvidarse de los detalles fuertemente restrictivos de la naturaleza de Coq, con un esfuerzo pequeño.

\chapter*{Abstract}

Meta-languages are an essential part of theorem provers. This meta-languages have to deal with several problems, hence they employ different mechanisms.
Particularly speaking, the meta-language \Mtac on Coq, utilizes monads as a means to add rich typed tactics to Coq.

This monadic programs are hard to develop. Partly, \textit{binding} monadic elements when Coq's type system uses dependent types is not simple.
Solving this involves great effort on developing useless code that is not related to the actual problem.

Using \Mtac we can create a new meta-program that can almost automatically generalize the signature of functions, allowing a seamless communication between monadic statements.

With this work, the programmer can focus on truly important tasks, forgetting the highly restrictable nature of Coq, with little effort.

\vfill

% \begin{otherlanguage}{ngerman}
% \pdfbookmark[1]{Zusammenfassung}{Zusammenfassung}
% \chapter*{Zusammenfassung}
% Kurze Zusammenfassung des Inhaltes in deutscher Sprache\dots
% \end{otherlanguage}

\endgroup

\vfill
