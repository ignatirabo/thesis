\chapter{Introducción}\label{ch:intro}

El uso de \emph{asistentes de prueba} es cada vez mayor.
Múltiples de ellos cuentan con metalenguajes de programación que permiten la automatización del proceso de desarrollo de pruebas.
Dentro del espectro de los asistentes de prueba, Coq \cite{CIC} es una las opciones más conocidas, siendo \mtac \cite{DBLP:journals/pacmpl/KaiserZKRD18} uno de los metalenguajes disponibles.
En \Mtac, el uso de mónadas nos permite generar lo que se denomina \emph{tácticas}, en este caso, tipadas a diferencia de las usuales.
Estas metafunciones monádicas utilizan operadores monádicos para poder reflejar cómputos a través de la concatenación de subcómputos.

El uso de estas funciones monádicas es complejo.
El caso en el que nos centraremos es cuando nuestros operadores monádicos no sean lo suficientemente generales para poder \emph{vincular} elementos monádicos.
En principio, estos problemas no surgirán necesariamente, pero si el desarrollador está utilizando tipos dependientes, utilizará cuantificadores para poder expresar estas dependencias. La problemática es que los operadores monádicos de \mtac no permiten tomar, ni retornar, elementos cuantificados.

Este trabajo podría haberse resumido a codificar vesiones más expresivas de los dos operadores monádicos, \lstinline{bind} y \lstinline{ret}.
Sin embargo, hemos preferido desarrollar una solución más general.
Lo que tenemos es una metafunción que toma a otras metafunciones como argumento y retorna ``versiones cuantificadas'' de las mismas, de manera casi automática.
De esta forma, los dos operadores a los que nos referimos, simplemente necesitan ser pasados como argumentos a nuestra nueva metafunción, que hemos denominado \lift.

% Qué es un asistente de prueba?
\section{Asistentes de prueba}

Históricamente, una prueba matemática es una sucesión de instrucciones, más específicamente, reglas de inferencia, en un lenguaje más formal que el natural y que, al ser aplicadas en orden, partiendo de una serie de hipótesis, pueden llegar a una conclusión, el teorema.
Si estas reglas son utilizadas correctamente y las hipótesis no se contradicen, entonces todo resultado al que lleguemos puede ser considerado verdadero. Efectivamente, estos pasos representan una prueba de dicho resultado.

Sin embargo, en las pruebas matemáticas no explicitamos cada regla que utilizamos, pues la lectura de la demostración resultaría muy tediosa y, claramente, no es necesario que sea extremadamente formal para que el lector pueda comprender la prueba.
Pero entonces, estas pruebas no tiene la rigurosidad absoluta que desearíamos, podemos decir que estas siguen siendo, en cierto punto, informales.
En este contexto, es claro que los lectores no pueden dar un veredicto exacto sobre el valor de verdad de un teorema.

Tenemos la suerte de que las computadoras son muy adecuadas para este tipo de trabajos rigurosos. De aquí surge la motivación de desarrollar herramientas para verificar estos resultados matemáticos, pero que también puedan asistir al desarrollador en el proceso. Algunas de estas herramientas son Coq \cite{CIC}, Isabelle \cite{DBLP:books/sp/NipkowPW02}, Agda \cite{DBLP:conf/tphol/BoveDN09}, Lean \cite{DBLP:conf/cade/MouraKADR15} y HOL4 \cite{DBLP:conf/tphol/SlindN08}.

% Qué es Coq?
\section{Coq}

Coq es uno de los asistentes de prueba más utilizados.
Este cuenta con numerosos casos de éxito, tanto en matemática \cite{DBLP:conf/ascm/Gonthier07}, como en computación \cite{DBLP:journals/pacmpl/0002JKD18}.

Uno de los aspectos más importantes de Coq es que las pruebas se codifican a través de la concatenación de \emph{tácticas}.
Supongamos que tenemos un teorema a probar.
Pensaremos este teorema como una meta o un objetivo.
A través del uso de tácticas, modificaremos la meta, también llamada \emph{goal}.
Las tácticas generarán nuevas metas, que representarán pasos intermedios de la demostración.
Por ejemplo, podemos pensar la inducción matemática como una táctica, la cual tomará nuestra meta actual, generando dos nuevas metas: el caso base y el paso inductivo. En este caso, de una meta generamos dos nuevas.
El trabajo para desarrollar una prueba consiste en utilizar estas tácticas para reducir nuestra meta hasta que esta sea verdadera.

El aspecto más importante de las tácticas, al menos en nuestro caso, es que el programador puede desarrollar sus propias tácticas con el objetivo de ser asistido a la hora de escribir la prueba.
Estás tácticas son, por defecto, definidas en el metalenguaje Ltac \cite{DBLP:conf/lpar/Delahaye00}.

Existen múltiples tipos de tácticas que intentan diferentes cosas, e inclusive, con diferentes filosofías.
Un ejemplo es la táctica CoqHammer \cite{DBLP:journals/jar/CzajkaK18}, una de las más conocidas en Coq. Esta tiene como objetivo demostrar todo automáticamente.
Aunque esta táctica pueda resultar muy poderosa, también conlleva muchos problemas, como mencionaremos más adelante.

% Qué es Mtac2?
\section{Mtac2}

El metalenguaje \Mtac \cite{DBLP:journals/pacmpl/KaiserZKRD18} tiene como objetivo reemplazar a Ltac.
En este sentido, es uno de tantos que intentan lo mismo: Ltac2~(Cap.~Ltac2 en Manual de Referencia de Coq 8.11 \cite{Coq}), Template-Coq \cite{DBLP:conf/itp/AnandBCST18}, Rtac \cite{DBLP:conf/esop/MalechaB16} y Coq-Elpi \cite{tassi:hal-01637063}.

Volviendo a CoqHammer, esta táctica presenta varias deficiencias.
En parte, esta táctica no es rápida: ejecutarla puede tardar múltiples horas, y aún así fallar.
Si todo funciona, no debemos preocuparnos.
Si falla, no tenemos certeza de por qué fue.
Más aun, podríamos haber utilizado la táctica de manera incorrecta y no saberlo, ya que las tácticas de Ltac no son tipadas.
Es decir, podemos intentar utilizar cualquier táctica en cualquier momento, pero solo funcionarán las tácticas adecuadas.
Por eso, una de las características más importantes de \mtac es que tiene tácticas tipadas.
Estas, solo pueden ser utilizadas en el momento en que la meta que queremos probar se aproxima a lo que nuestra táctica acepta.
Además nos asegura de qué forma será la meta resultante de la ejecución.
Por último, en el caso de fallar, tendremos certeza de cual fue la razón.

% No hablar de la solución

\section{Desarrollo del trabajo}
% Donde explicás, capítulo por capítulo lo que se hace. Es como una guía para leer el trabajo  donde se explica que se hace en cada capítulo, donde están los resultados y el resultado principal.

En el Capítulo \ref{ch:coq}, podemos encontrar una introducción a Coq, con todos los conceptos principales que utilizaremos más adelante.
Primero introduciremos los diferentes lenguajes que residen en Coq, donde dos de ellos serán de gran importancia: \textsc{Ltac} y \textsc{Gallina}.
Veremos como las tácticas de Coq, definidas en Ltac, nos permiten codificar pruebas matemáticas.
En cuanto a Gallina, lo utilizaremos para poder definir nuestras funciones y tipos.
En esta introducción, cubriremos tipos inductivos, recursivos y dependientes.
Utilizaremos $\Sigma$-tipos para poder expresar dependencias en los tipos y definir funciones que ofrezcan más certezas al depender de pruebas.
Además, discutiremos cuales son las consecuencias de que Coq sea un lenguaje puro.

En el Capítulo \ref{ch:mtac2} encontraremos una introducción a \Mtac.
Primero, introduciremos el concepto de mónadas, ya que \mtac define una mónada para reflejar efectos secundarios.
Hablaremos de la creación de metaprogramas, cubriendo las versiones monádicas de pattern matching y recursión.
Luego, discutiremos la falta de garantías que estos metaprogramas proveen en comparación con programas puros de Coq.
Finalmente, definiremos \emph{telescopios} en \Mtac. Esta signatura es la que nos permitirá diseñar nuestra metafunción \lift.

En la segunda parte de este trabajo encontraremos tres capítulos:

\begin{itemize}
 \item En el Capítulo \ref{ch:motivacion} expondremos con mayor precisión el problema que estamos resolviendo con una función modelo.
 Esta función tiene el problema de que no podrá ser parametrizada automáticamente.
 Ahí entrará en juego \lift, generando las metafunciones necesarias para que la función modela pueda ser parametrizada.
 \item En el Capíutlo \ref{ch:lift} analizaremos a la metafunción \lift de manera conceptual.
 Analizaremos las signaturas de las metafunciones originales, y llegaremos a la signatura deseada, dejando en evidencia nuestra necesidad de generalizarlas.
 Nos concentraremos en comprender la lógica de la solución.
 \item En el Capítulo \ref{ch:technical} haremos un estudio en profundidad de la función \lift.
 Comenzaremos estudiando el tipo auxiliar \lstinline{TyTree}, el cual nos permitirá describir las signaturas de nuestras funciones con un tipo inductivo.
 Luego, definiremos múltiples funciones y tipos auxiliares que serán suficientes para poder definir \lift.
 Y cerraremos este capítulo con dos ejecuciones de \lift, paso por paso.
 \item En el Apéndice \ref{ch:apendice}, encontraremos la definición de todas las funciones y tipos utilizados en \lift, junto con la definición completa de \lift.
\end{itemize}