\chapter{Motivación}\label{ch:motivacion}

% Se asume que se explicó que es Mtac2 en la introducción y todas las cosas que
% se pueden hacer.
\textsc{Mtac2} nos permite definir funciones monádicas. Estas cuentan con ciertas ventajas.
La siguiente función calcula el máximo de una lista de números \lstinline{nat : Set} y utiliza \lstinline{mtmmatch} para hacer un analisis sintáctico más profundo del que se puede realizar en \lstinline{match}. % TODO: hablar de mtmmatch en la intro.

\begin{lstlisting}[float=h,frame=tb,caption={Teorema y prueba en Coq},label=lst:list_max_nat]
Definition list_max_nat :=
  mfix f (l: list nat) : l <> nil -> M nat :=
    mtmmatch l as l' return l' <> nil -> M nat with
    | [? e] [e] =m> fun nonE => M.ret e
    | [? e1 e2 l'] (e1 :: e2 :: l') =m> fun nonE =>
      let m := Nat.max e1 e2 in
      f (m :: l') cons_not_nil
    | [? l' r'] l' ++ r' => (* cualquier cosa *) 
    end.
\end{lstlisting}

El último caso de \lstinline{mtmmatch} analiza una expresión que no es un constructor del tipo inductivo \lstinline{list}. Por eso es necesario utilizar el \lstinline{match} monádico.
con una función, y no un constructor, es imposible implementar esto sin Mtac2. Notar que tampoco podemos utilizar \lstinline{mmatch} ya que el tipo que retornamos terminos de tipo \lstinline{l' <> nil -> M nat}.

Ahora supongamos que deseamos parametrizar $\nat$ y tener una función que acepte
múltiples conjuntos. Sea
\begin{lstlisting}
Definition max (S: Set) : M (S -> S -> S) :=
  mmatch S in Set as S' return M (S' -> S' -> S') with
  | nat => M.ret Nat.max
  end.
\end{lstlisting}
la función que retorna la relación máximo en un conjunto $S$.
A primera vista nuestra idea podría fucionar, es decir, conceptualmente no es incorrecta.

\begin{lstlisting}
Definition list_max (S: Set)  :=
  max <- max S; (* error! *)
  mfix f (l: list S) : l' <> nil -> M S :=
    mtmmatch l as l' return l' <> nil -> M S with
    | [? e] [e] =m> fun nonE=>M.ret e
    | [? e1 e2 l'] (e1 :: e2 :: l') =m> fun nonE =>
      m <- max e1 e2;
      f (m :: l') cons_not_nil
    end.
\end{lstlisting}

Al intentar que Coq interprete la función veremos que esta función no tipa.
Esto es debido a la signatura de \lstinline{bind}.
Nuestro \lstinline{mfix} no puede unificarse a \lstinline{M B}, ya que tiene
tipo \lstinline{f : forall (l: list S)  l' <> nil -> M S}.

\begin{lstlisting}
bind : forall A B, M A -> (A -> M B) -> M B.
\end{lstlisting}
% En la motivación usar el ret lifteando antes del fun/lambda y el bind lifteado para el max.

Solucionar esta situación específica no es un problema.
Una alternativa es introducir los parámetros de la función y beta-expandir la definición del fixpoint.
Otra es codificar un nuevo bind que tenga el tipo necesario.
El problema será que ambas soluciones son específicas al problema, entonces en cada situación debemos volver a implementar alguno de estos recursos.

Es por eso que nuestro proyecto es la codificación de un nuevo meta-programa \lift{} que automaticamente puede generalizar meta-programas con las dependencias necesarias para que sea utilizado en el contexto.
En nuestro ejemplo, con \lift{} podemos generalizar \lstinline{bind} consiguiendo un nuevo meta-programa que se comporta como la función original pero con una signatura distinta, permitiendo su uso.
