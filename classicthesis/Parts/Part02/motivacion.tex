\chapter{Motivación}\label{ch:motivacion}

% Se asume que se explicó que es Mtac2 en la introducción y todas las cosas que
% se pueden hacer.
\textsc{Mtac2} nos permite definir funciones monádicas. Estas cuentan con ciertas ventajas.
La siguiente función calcula el máximo de una lista de números \lstinline{nat : Set} y utiliza \lstinline{mtmmatch} para hacer un análisis sintáctico, a diferencia de \lstinline{match}.

\begin{lstlisting}[frame=tb,caption={Función \lstinline{list_max_nat}},label=lst:list_max_nat]
Definition list_max_nat :=
  mfix f (l: list nat) : l <> nil -> M nat :=
    mtmmatch l as l' return l' <> nil -> M nat with
    | [? e] [e] => fun P => M.ret e
    | [? e1 e2 l'] (e1 :: e2 :: l') => fun P =>
      let x := Nat.max e1 e2 in
      f (x :: l') _
    | [? l' r'] app l' r' => fun P =>
      match app_not_nil l' r' P with
      | inl P' => f l' P'
      | inr P' => f r' P'
      end
    end.
\end{lstlisting}
% TODO: Beta dice que hay que usar disjoint sum "x + y" y no el \/ (or).

El último caso de \lstinline{mtmmatch} analiza una expresión que no es un constructor del tipo inductivo \lstinline{list}. Por eso, es necesario utilizar el \lstinline{match} monádico.
Notar que tampoco podemos utilizar \lstinline{mmatch} ya que deseamos retornar términos de tipo \lstinline{l' <> nil -> M nat}.

Ahora, supongamos que deseamos parametrizar $\nat$ y utilizar una función que acepte múltiples conjuntos de nuestro interés.

\begin{lstlisting}[frame=tb,caption={Función \lstinline{max}},label=lst:max]
Definition max (S: Set) : M (S -> S -> S) :=
  mmatch S in Set as S' return M (S' -> S' -> S') with
  | nat => M.ret Nat.max
  end.
\end{lstlisting}

Sea \lstinline{max} en \ref{lst:max}, la función que retorna la relación máximo en un conjunto $S$.
A primera vista, nuestra idea podría funcionar, es decir, intuitivamente no es ilógica.

\begin{lstlisting}[frame=tb,caption={Función \lstinline{list_max}},label=lst:list_max]
Definition list_max (S: Set) :=
  max <- max S; (* error! *)
  mfix f (l: list S) : l <> nil -> M S :=
    mtmmatch l as l' return l' <> nil -> M S with
    | [? e] [e] => fun P => M.ret e
    | [? e1 e2 l'] (e1 :: e2 :: l') => fun P =>
      m <- max e1 e2;
      f (m :: l') _
    | ...
    end.
\end{lstlisting}

Al intentar que Coq interprete la función veremos que esta función no tipa.
Esto es debido a que \lstinline{bind} no tiene la signatura necesaria.

\begin{lstlisting}
bind : forall A B, M A -> (A -> M B) -> M B
\end{lstlisting}

Nuestro \lstinline{mfix} no puede unificarse a \lstinline{M B}, ya que su tipo es una implicación: \lstinline{(f : forall (l: list S),  l' <> nil -> M S)}.
% Es decir, tenemos \lstinline{A = (S -> S -> S)} y \lstinline{B = forall (l: list S)  l' <> nil -> M S} entonces no podemos utilizar \lstinline{bind}.
% TODO: En la motivación usar el ret lifteando antes del fun/lambda y el bind lifteado para el max.

Solucionar esta situación específica no es un problema.
Una alternativa es introducir los parámetros de la función y beta-expandir la definición del fixpoint.
Otra, es codificar un nuevo \lstinline{bind} que tenga el tipo necesario.
El problema será que ambas soluciones son específicas al problema, entonces, en cada situación, debemos volver a implementar alguno de estos operadores.

En nuestra solución, suplantamos \lstinline{bind} y \lstinline{ret} por \lstinline{bind^} y \lstinline{ret^}, respectivamente. Llamaremos a estas funciones las versiones \emph{lifteadas} de las originales. Estas poseen el tipo necesario.

Nuestro proyecto es la codificación de un nuevo metaprograma \lift que puede generalizar metaprogramas con las dependencias necesarias para que sea utilizado en el contexto.
En nuestro ejemplo, utilizando \lift podemos generalizar \lstinline{bind} y \lstinline{ret}, consiguiendo nuevos metaprogramas que se comportan como los originales pero con una signatura distinta, permitiendo su uso.

\begin{lstlisting}[frame=tb,caption={Función \lstinline{list_max} lifteada},label=lst:list_max_lifted]
Definition list_max_lifted (S: Set) :=
max <^- max S; (* notacion para bind^ *)
mfix f (l: list S) : l <> nil -> M S :=
    mtmmatch l as l' return l' <> nil -> M S with
    | [? e] [e] => ret^ (fun _ => e)
    | [? e1 e2 l'] (e1 :: e2 :: l') => fun P =>
      m <- max e1 e2;
      f (m :: l') _
    | ...
    end.
\end{lstlisting}

Veremos que la función \lift requiere pequeño esfuerzo para ser utilizada.