\chapter{Lift}\label{ch:lift}

Denominamos \textsc{Lift} al meta-programa desarrollado en este trabajo.
Esta función tiene la tarea de agregar dependencias a otros meta-programas de manera cuasi automática. Veremos que solo depende de un telescopio, el que en principio generaremos nosotros.

\textsc{Lift} se basa en analizar los tipos de las funciones y modificarlos añadiendo dependencias triviales en los tipos que se encuentran bajo la mónada, generando así nuevos meta-programas más generales.
Cuando digamos ``dependencias triviales'' nos referiremos a valores en el tipo que son dependientes pero no tienen ningún valor real, es decir, no influyen en el resultado de la ejecución.
Aunque esto pueda parecer inutil, su justificación es que hacer estos agregados efectivamente generan funciones más generales que pueden ser utilizadas en una amplia cantidad de situaciones. 
El \emph{lifteo} de funciones será trabajado en este capítulo a través de ejemplos.

\section{Definiendo objetivos}

Una de las funciones monádica más simples es \lstinline{ret : \forall A, A -> M A}, uno de los operadores monádicos.

\begin{lstlisting}[float=h,frame=tb,caption={Signatura de ret},label=lst:ret]
ret : forall A, A -> M A
\end{lstlisting}

Como vemos en la signatura de \lstinline{ret}, solo puede devolver expresiones \lstinline{M A} donde no es posible que haya dependencias en \lstinline{A}. Esto hace que en una serie de situaciones no sea posible usar esta función.
% TODO: linkear con motivación y lift_max.

% Supongamos que nos interesa tener
% \begin{lstlisting}
% ret^ : \forall (A : nat -> nat -> Type), (\forall n n', A n n') -> (\forall n n', M (A n n'))
% \end{lstlisting}

Ahora, nuestro trabajo es poder definir un telescopio que se amolde a la información que necesitamos agregar.
En este caso el telescopio es el siguiente: \lstinline{t := [n : nat ;> n' : nat]$_t$} dado que queremos que \lstinline{A} dependa de dos $\nat$. Efectivamente \lstinline{ret^ := lift ret t}. Este ejemplo es simple porque la signatura de \lstinline{ret} no contiene funciones. Pero en ese caso, estudiemos que sucede con \lstinline{bind}.

Bind es el otro operador monádico y su signatura presenta un problema. Esto es debido a que su signatura contiene una función y esta depende de los argumentos \lstinline{A} y \lstinline{B} que se encuentran bajo la mónada en diferentes partes del tipo.

\begin{lstlisting}[float=h,frame=tb,caption={Signatura de bind},label=lst:bind]
bind : forall A B, M A -> (A -> M B) -> M B
\end{lstlisting}

Volviendo a ref(motivacion) vimos que nuestro bind no es suficientemente general. El candidato a ser \lstinline{A} es \lstinline{S -> S -> S}, pero para \lstinline{B} dijimos \lstinline{l' <> nil -> M S} aunque esto es incorrecto. Lo que verdaderamente necesitamos es exprezar en un telescopio que deseamos que \texttt{S} dependa de \lstinline{l' <> nil}, es decir, \lstinline{S : \forall (p : l' <> nil), Type}. Claramente cuantificar esta parte del tipo es más fuerte que una dependencia, pero a nuestros fines no es un problema.

Ya con esta idea, podemos ver que el tipo que queremos que \lstinline{bind} retorne es \lstinline{\forall (p : l' <> nil), M (S p)}. Y finalmente llegamos al siguiente resultado.

\begin{lstlisting}[float=h,frame=tb,caption={Nueva signatura de bind},label=lst:bind_motiv]
bind^ : \forall (A B : \forall (p : l' <> nil), Type),
  (\forall p, M (A p)) -> )(\forall p, A p -> B p) -> (\forall p, M (B p)).
\end{lstlisting}

\section{Creando telescopios}

Como vimos anteriormente, para liftear una función necesitamos definir un telescopio. Este es el que indicará los argumentos dependientes que debamos agregar a nuestra función.
% TODO: hablar de la futura creación automática.

Para el caso de \ref{lst:list_max_lifted} debemos añadir una única dependencia de tipo \lstinline{l' <> nil}. Por ese motivo necesitaremos un telescopio de largo uno.

\begin{lstlisting}[float=h,frame=tb,caption={Nueva signatura de bind},label=lst:list_max_tele]
tele_motiv : MTele := [p : l' <> nil]$_t$
\end{lstlisting}

Cabe aclarar que \lstinline{l'} es una variable ya definida en el contexto, y por eso nos referimos a esa lista particular y no a cualquier otra.

En otros casos definiremos telescopios con más argumentos y estos argumentos pueden ser dependientes. Por ejemplo, si \lstinline{l'} no estuviera definida previamente, necesitariamos que nuestro telescopio sea algo como  \lstinline{t : MTele := [l' : list S ;> p : l' <> nil]$_t$}. Y así sucesivamente.

% TODO: referenciar trabajo futuro
En ref(futuro) discutiremos sobre la capacidad de generar estos telescopios de manera automática analizando el tipo deseado de nuestra función lifteada.

\section{El resultado}

La función \textsc{Lift} retorna una tupla dependiente con el tipo de la función lifteada y la función lifteada. Ya ahora con nuestro telescopio \lstinline{t}, extraer la nueva función es simple.

\graffito{\lstinline{mprojT2} nos permite extraer el segundo argumento de una tupla dependiente en \textsc{Mtac2}}
\begin{lstlisting}[float=h,frame=tb,caption={Lifteando \lstinline{ret} y \lstinline{bind}},label=lst:lift1]
ret^ := mprojT2 (lift ret t)
bind^ := mprojT2 (lift bind t)
\end{lstlisting}

Finalmente nuestro programa quedaría de la siguiente forma.

\begin{lstlisting}[float=h,frame=tb,caption={Lifteando \lstinline{ret} y \lstinline{bind}},label=lst:lift1]
Definition list_max (S: Set)  :=
  max <^- max S; (* error! *)
  mfix f (l: list S) : l' <> nil -> M S :=
    mtmmatch l as l' return l' <> nil -> M S with
    | [? e] [e] =m> ret^ e
    | [? e1 e2 l'] (e1 :: e2 :: l') =m> fun nonE =>
      m <- max e1 e2;
      f (m :: l') cons_not_nil
    end.
\end{lstlisting}

\iffalse
% TODO: Parte antigua, revisar si hay algo útil.
Supongamos que queremos utilizamos el telescopio \lstinline{t := [T : Type ;> l : list T]$_t$} donde \lstinline{list T} es el tipo de las listas con elementos de tipo $T$.
Al momento de liftearlo, no es obvio cual debería ser el resultado, así que veamoslo.

\begin{lstlisting}
bind^ : \forall A B : \forall T : Type, list T -> Type,
         (\forall (T : Type) (l : list T), M (A T l)) ->
         (\forall (T : Type) (l : list T), A T l -> M (B T l)) ->
         (\forall (T : Type) (l : list T), M (B T l))
\end{lstlisting}

Lo más imporante es que si observamos la definición de esta función es sumamente simple.

\begin{lstlisting}
fun (A B : \forall T : Type, list T -> Type)
    (ma : \forall (T : Type) (l : list T), M (A T l))
    (f : \forall (T : Type) (l : list a), A T l -> M (B T l))
    (T : Type) (l : list T) =>
    bind (A T l) (B T l) (ma T l) (f T l)
\end{lstlisting}

Esta definición es efectivamente la que buscabamos y funciona perfectamente.
El otro aspecto que seguiremos observando es que Lift no genera información innecesaria en la función destino.
\fi