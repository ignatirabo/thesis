\chapter{Lift}\label{ch:lift}

Denominamos \textsc{Lift} al metaprograma desarrollado en este trabajo.
Esta función tiene la tarea de agregar dependencias a otros metaprogramas de manera cuasi automática. Veremos que solo depende de un telescopio, el que en principio generaremos nosotros.

\textsc{Lift} se basa en analizar los tipos de las funciones y modificarlos añadiendo dependencias triviales en los tipos que se encuentran bajo la mónada, generando así nuevos metaprogramas más generales.
Cuando digamos ``dependencias triviales'' nos referiremos a valores en el tipo que son dependientes pero no tienen ningún valor real, es decir, no influyen en el resultado de la ejecución.
Aunque esto pueda parecer inútil, su justificación es que hacer estos agregados, efectivamente genera funciones más generales que pueden ser utilizadas en una amplia cantidad de situaciones. 

Esta generalización hace uso de telescopios para poder especificar las dependencias que se desea agregar.
Primero observaremos la situación desde un punto de vista más bien intuitivo, y finalizaremos concretizando los telescopios.

\section{Definiendo objetivos}

Una de las funciones monádica más simples es \lstinline{ret}, uno de los operadores monádicos.

\begin{lstlisting}[frame=tb,caption={Signatura de \lstinline{ret}},label=lst:ret]
ret : forall A, A -> M A
\end{lstlisting}

Como vemos en la signatura de \lstinline{ret}, solo puede devolver expresiones \lstinline{(M A)} donde no es posible que el retorno esté cuantificado, es decir, no puede retornar una función.
Si tomamos \lstinline{list_max_lifted} \ref{lst:list_max_lifted}, la función \lstinline{ret^} toma un valor de tipo \lstinline{(l' <> nil -> S)} y retorna \lstinline{l' <> nil -> M S}, que también puede ser expresado como \lstinline{forall (p : l' <> nil), M S}.
Notemos que en \lstinline{list_max_lifted}, el argumento de \lstinline{ret^} no utiliza el valor introducido en la función.

En principio, el tipo \lstinline{A : Type} de \lstinline{ret} será reemplazado por un nuevo tipo que tome la prueba: \lstinline{(A : l' <> nil -> Type)} y naturalmente podemos observar como queda el resto. 

\begin{lstlisting}[float=h,frame=tb,caption={Signatura deseada de \lstinline{ret}},label=lst:ret_motiv]
ret^ : \forall (A : l' <> nil -> Type), (\forall p, A p) -> (\forall p, M (A p))
\end{lstlisting}

Ahora estudiemos que sucede con \lstinline{bind}.

El operador \lstinline{bind} es la otra función que debemos modificar, pero su signatura presenta un problema más complejo que \lstinline{ret}. Esto es debido a que su signatura contiene una función y esta depende de los argumentos \lstinline{A} y \lstinline{B}, que se encuentran bajo la mónada en diferentes partes del tipo.

\begin{lstlisting}[frame=tb,caption={Signatura de \lstinline{bind}},label=lst:bind]
bind : forall A B, M A -> (A -> M B) -> M B
\end{lstlisting}

Volviendo a \lstinline{list_max} \ref{lst:list_max}, pudimos notar que \lstinline{bind} no es suficientemente general.
En \ref{lst:list_max_lifted}, observamos que queremos bindear \lstinline{max S} con nuestro fixpoint. La signatura de \lstinline{max S} es \lstinline{M (S -> S -> S)}.
La signatura del fixpoint es \lstinline{forall (l : list S), l <> nil -> M S}, por lo tanto, es lo que debe retornar nuestro \lstinline{bind}.

Ahora intentemos armar la signatura del nuevo \lstinline{bind}. Tenemos que los tipos \lstinline{A} y \lstinline{B} que toma \lstinline{bind} deben ser reemplazados por otros más generales. Si hacemos que estos dos tipos sean dependientes de los mismos argumentos podemos facilitar este razonamiento.
Entonces, si \lstinline{(A B : \forall l (p : l <> nil), Type)}, podemos reescribir la signatura.

\begin{lstlisting}[frame=tb,caption={Signatura deseada de \lstinline{bind}}]
bind^ : \forall (A B : \forall l (p : l <> nil), Type),
  \forall l p, M (A l p) ->
  \forall l p, (A l p -> M (B l p)) ->
  \forall l p, M (B l p)
\end{lstlisting}

\graffito{TODO: pegarle última leida}
En esta signatura podemos notar que el retorno de \lstinline{bind^} es, efectivamente, el tipo del punto fijo. También notamos que la función de \lstinline{A} en \lstinline{M B}, intuitivamente sigue siendo eso, solo que tomamos dos argumentos \lstinline{l} y \lstinline{p} los cuales \lstinline{A} y \lstinline{B} están compartiendo.
Ahora, es importante que al momento de utilizar \lstinline{bind^}, los dos tipos que toma estén bien definidos. Necesitamos que \lstinline{(B l p)} reduzca a \lstinline{S} y que \lstinline{(A l p)} reduzca a \lstinline{(S -> S -> S)}.

\section{Creando telescopios}

Como vimos anteriormente, para liftear una función necesitamos definir un telescopio. Este es el que indicará los argumentos dependientes que debamos agregar a nuestra función.

Como vimos recién, para \lstinline{bind^} en el caso de \ref{lst:list_max_lifted} debemos añadir dos dependecias a los tipos: \lstinline{l : list S} y \lstinline{p : l <> nil}.
Por ese motivo necesitaremos un telescopio de largo dos.
Pero dado que \lstinline{ret^} es utilizado dentro del fixpoint, \lstinline{l} ya se encuentra en el contexto y no es necesario añadirlo como un dato al telescopio. Entonces necesitaremos dos telescopios.

\begin{lstlisting}[frame=tb,caption={Telescopio para \lstinline{list_max}},label=lst:list_max_tele]
m : MTele := [l : list S ;> p : l <> nil]$_t$
n : MTele := fun l => [p : l <> nil]$_t$
\end{lstlisting}

No es un problema que \lstinline{n} dependa de \lstinline{l}. Podemos crear una lista cualquiera al momento de liftear.

En \ref{sc:futuro} discutiremos sobre la capacidad de generar estos telescopios de manera automática analizando el tipo deseado de nuestra función lifteada.

\section{El resultado}

La función \lift retorna una tupla dependiente con el tipo de la función lifteada y la función lifteada. Ya ahora con nuestro telescopio \lstinline{m}, extraer la nueva función es simple.

\graffito{\lstinline{mprojT2} nos permite extraer el segundo argumento de una tupla dependiente en \textsc{Mtac2}}
\begin{lstlisting}[float=h,frame=tb,caption={Lifteando \lstinline{ret} y \lstinline{bind}},label=lst:lift1]
bind^ := mprojT2 (lift bind m)
ret^ := mprojT2 (lift ret (n l))
\end{lstlisting}

Finalmente nuestro programa quedaría de la siguiente forma.

\begin{lstlisting}[float=h,frame=tb,caption={Lifteando \lstinline{ret} y \lstinline{bind}},label=lst:lift1]
Definition list_max_lifted (S: Set) :=
max <^- max S; (* notacion para bind^ *)
mfix f (l: list S) : l' <> nil -> M S :=
    mtmmatch l as l' return l' <> nil -> M S with
    | [? e] [e] => ret^ (fun _ => e)
    | [? e1 e2 l'] (e1 :: e2 :: l') => fun P =>
      m <- max e1 e2;
      f (m :: l') _
    | [? l' r'] app l' r' => fun P =>
      match app_not_nil l' r' P with
      | inl P' => f l' P'
      | inr P' => f r' P'
      end
    end.
\end{lstlisting}

En la próxima parte analizaremos cómo funciona \lift desde adentro.