\chapter{Conclusiones y trabajo futuro}\label{ch:conclusion}

\section{Conclusiones}

En la parte \ref{pt:introduccion} estudiamos al asistente de pruebas Coq y el meta-lenguaje de programación \mtac. Vimos como los tipos dependientes son sumamente importantes y como en \mtac se utilizan mónadas para escribir los meta-programas. 
Luego, en la parte \ref{pt:lift} comprendimos el problema que se puede generar al utilizar los operadores monádicos de \mtac y un uso fuerte de tipos dependientes. Analizamos las signaturas de las funciones y se concluyó en que se necesitaban nuevas funciones más generales.
Finalmente, se desarrolló el meta-programa \lift que puede generalizar otros meta-programas de manera cuasi automática. Para hacer esto realizamos un análisis por caso sobre la signatura de las funciones, y utilizamos telescopios para expresar los argumentos dependientes que se desea agregar.

\section{Trabajo futuro}

En algún punto se planea añadir \lift a \mtac como un feature por defecto. Pero principalmente el interés cae en implementar notación inteligente que pueda deducir telescopios. Como vimos en \ref{ch:motivacion}, las funciones \lstinline{bind} y \lstinline{ret} son funciones que queremos liftear, y probablemente con mayor frecuencia que otras. Con Coq podemos inferir el tipo necesario de estos dos operadores (y otros) y de esta manera podemos generar los telescopios, es decir, liftear funciones de manera completamente automática.
La notación es la siguiente.

\begin{lstlisting}
a <^- b (* bind de a en b pero con un bind lifteado *)
a;^;b (* equivalente a arriba pero sin conservar computo *)
g^ (* función cualquiera lifteada *)
\end{lstlisting}

Parte de esta notación ya ha sido desarrollada.