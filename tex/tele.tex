% En este archivo definimos la verdadera salsa!
\section{Telescopios}

En Mtac2 llamamos \textit{telescopios} a una estructura de datos inductiva
que permite expresar una cantidad arbitraria de tipos.

\begin{coqe}
Inductive MTele : Type :=
| mBase : MTele
| mTele {X : Type} (F : X -> MTele) : MTele.
\end{coqe}

El tipo \coqi{MTele} crea una cadena de abstracciones que toma valores de tipos específicos.

Los telescopios, junto con las funciones que lo acompañan será la base para nuestro
trabajo.

Eeste tipo puede pensarse en varias jerarquías.
La primera sería el telescopio mismo.
Un ejemplo puede ser:

\begin{coqe}
Let m := @mTele nat (fun _ : nat => mBase)
\end{coqe}

Este telescopio lleva un único tipo, \coqi{nat}.
Luego existen una infinita cantidad de tipos que dependen de \coqi{m}, para
continuar el ejemplo elegimos uno.

\begin{coqe}
Let A : MTele_Sort SProp m := fun x => x = x.
\end{coqe}

Finalmente, nos interesa un valor de ese tipo, es decir, una prueba.

\begin{coqe}
Let a : MTele_val (MTele_C SProp SProp M A) := fun x => ret (eq_refl).
\end{coqe}
