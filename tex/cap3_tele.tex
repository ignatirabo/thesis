\section{Telescopios}

En Mtac2 llamamos \textit{telescopios} a una estructura de datos inductiva
que permite expresar una secuencia de tipos posiblemente dependientes de largo arbitrario.

\begin{lstlisting}
Inductive MTele : Type :=
| mBase : MTele
| mTele {X : Type} (F : X -> MTele) : MTele.
\end{lstlisting}

% TODO: Para que se usa MTele en general?

El tipo \lstinline{MTele} crea una cadena de abstracciones.
Este se codifica a través de funciones, lo que permite que sean dependientes, es decir, un telescopio puede tener elementos que dependan de elementos anteriores.

Los telescopios, junto con las funciones que lo acompañan, serán claves a la hora de poder expresar nuestro problema y nuestra solución.

Los telescopios no se caracterizan por su simpleza.
Para comenzar a estudiarlos podemos pensarlos en jerarquías.
La primera jerarquía serían los telescopios en sí, elementos de tipo \lstinline{MTele}. Definiremos la siguiente notación para poder referirnos a estos de manera más simple.

\begin{lstlisting}
mBase $\equiv$ [$\;$]$_t$
mTele (fun T : Type => mTele R : T -> Type) $\equiv$ [T : Type;> R : (T -> Type)]$_t$
mTele (fun T : Type => mTele t : T) $\equiv$ [T : Type;> t : T]$_t$
\end{lstlisting}

Las otras jerarquías entran en juego cuando utilizamos las distintas funciones telescopicas.

La principal función es \lstinline{MTele_Sort}.
Dado un telescopio \lstinline{t} de largo $n$ y un \lstinline{s : Sort}, esta función computa \lstinline{\forall x$_1$ ... x$_n$, s}, es decir, retorna el tipo expresado por ese telescopio y el \lstinline{Sort} dado.
% TODO: Explicar que es Sort y stype_of
Utilizaremos \lstinline{MTele_Ty} y \lstinline{MTele_Prop} para expresar \lstinline{MTele_Sort Type$_s$} y \lstinline{MTele_Sort Prop$_s$} respectivamente. Elementos de este tipo serán de la segunda jerarquía.

También debemos poder referirnos a valores de estos tipos telescópicos. Esto signfica que dado un elemento de tipo \lstinline{MTele_Sort s t}, podamos conseguir un elemento de tipo \lstinline{s}. En terminos matemáticos, \lstinline{fun x$_1$ ... x$_n$ => T x$_1$ ... x$_n$} con \lstinline{T : MTele_Sort s t}. Consideraremos que estos elementos pertenecen a la tercera y última jerarquía.

% TODO: faltaría hablar de MTele_C que es el que nos permite definir MFA.
Deseriamos que este fuera el final de los telescopios, pero nos falta el último ingrediente que nos proveerá del poder que buscamos. Con \lstinline{MTele_In} podemos ganar acceso a múltiples tipos y valores telescopios al mismo tiempo, así siendo capaces de computar un nuevo tipo telescópico.
% TODO: más sobre esto, no explico nada. Se puede rellenar con muchos ejemplos?