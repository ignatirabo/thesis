% Parte vieja ?
\subsection{MFA}

Para seguir trabajando, debemos poder representar los tipos monádicos que nos
interesan.
Para esto definiremos \lstinline{MFA}.

\begin{lstlisting}
  Definition MFA {m} (T : MTele_Ty m) := (MTele_val (MTele_C SType SProp M T)).
\end{lstlisting}

Dado un telescopio \lstinline{m}, con $n$ tipos anidados y un tipo \lstinline{T} de
\lstinline{m}, \lstinline{MFA T} representa \lstinline{forall t1 ... tn, M (T t1 ... tn)}


\subsection{Bind}

% TODO Definir todas las cosas de Telescopios! tele.te

% Se asume que se han explicado todas las cosas de MTele.v

\begin{figure}[h]
  \centering
  \begin{lstlisting}
bind : forall A B : Type, M A -> (A -> M B) -> M B
mbind : forall m : MTele, A B : MTele_Ty m, MFA A -> (A -> MFA B) -> MFA B
  \end{lstlisting}
  \caption{Signaturas varias}
  \label{fig:bind}
\end{figure}

Para comenzar a estudiar el problema es mejor centrarse en casos más simples que
podamos razonar. La primera función interesante que podemos \textit{liftear} es
\lstinline{bind} \ref{fig:bind}. 

Es necesario poder entender cual es nuestro objetivo y decidir exactamente qué
buscamos modificar de la función. No existe una forma correcta de pensar el
tipo, sólo la que nos sirva. En nuestro caso, la idea más simple podemos verla
en \ref{fig:bind}.

En este caso, \lstinline{(A -> MFA B)} es una función que podemos pensar tiene un
tipo equivalente a \lstinline{(A -> forall t1 ... tn, M (B t1 ... tn))}. Otra forma
de pensarlo es con \lstinline{MTele_In} y \lstinline{accessor}. De esta forma, podemos
expresar \lstinline{forall t1 ... tn, (A -> M (B t1 ... tn))}. En la figura
\ref{fig:mbind} se puede observar el último caso. 

Ahora es momento de definir nuestra nueva función. El primer punto importante es
que en el caso de que \lstinline{m} sea vacío, \lstinline{mbind} se debe comportar justo
como \lstinline{bind}. El verdadero desafío está en la recursión. Dada la naturaleza
recursiva de los telescopios, cada paso recursivo se trata de pelar un tipo,
como una cebolla.

\begin{figure}[h]
\centering
\begin{lstlisting}
Fixpoint mbind {m : MTele} : forall {A B : MTele_Ty m},
  MFA A ->
    (MTele_val
      (MTele_In SType (fun a => let A' := a.(acc_sort) A in
                                let B' := a.(acc_sort) B in
                                (A' -> M B')
    ))) -> MFA B :=
  match m with
  | mBase =>
    fun A B ma f => @bind A B ma f
  | @mTele X F =>
      fun A B ma f x => @mbind (F x) (A x) (B x) (ma x) (f x)
  end.
\end{lstlisting}
\caption{El programa \lstinline{mbind}}
\label{fig:mbind}
\end{figure}

Lo importante de esta definición es que funciona para cualquier telescopio y
cualquier tipo \lstinline{A} y \lstinline{B}. Claramente, nos interesa que \lstinline{B}
efectivamente sea dependiente de todos los argumentos de \lstinline{m}, mientras
tanto, para \lstinline{A} no es necesariamente importarte, dado que el valor de
retorno no lo menciona.

\subsection{La Heurística}
% TODO: Hacer un estilo de reglas enumeradas para cada caso.
% TODO: Después de esto tengo que mencionar como se hace esto, entre eso que es
% RETURN y que es UNCURRY.

En el caso de \lstinline{mbind}, nosotros decidimos cual sería el tipo y adaptamos el
código en función de este. Pero nuestro objetivo va más allá. Queremos que
cualquier función sea automáticamente \textit{lifteada}. Esto nos obliga a
diseñar al programa a través de analisis de tipos, según yo una
\textbf{heurística} que determine el tipo final.

Podriamos tratar de definir \lstinline{lift} primero o podríamos tratar de definir el
resultado de cada tipo.

A continuación haremos un análisis de como reemplazar cada parte del tipo. Es
importante notar que el orden en que esto se define puede cambiar el resultado
final, se debe leer este listado asumiendo que se matchea de manera secuencial.
Esto se debe a que varios tipos resultan más generales que otros. A nosotros nos
interesa poder dividirlos de esta forma por conveniencia.

\begin{enumerate}
% X
\item \lstinline{tyTree_base X}: este es uno de los casos bases, no debemos realizar
  cambio alguno.
% M X
\item \lstinline{tyTree_M X}: este caso solo va a ser utilizado con el tipo de
  retorno del programa. Se divide en dos casos: el primero refleja \lstinline{M X}
  donde \lstinline{X} fue reemplazado por \lstinline{lift}, en el otro caso \lstinline{X} no
  fue reemplazado y por lo tanto lo llamaremos el caso constante.
  \begin{itemize}
  \item \lstinline{tyTree_M (apply_sort A U)}: 
  \item \lstinline{tyTree_M X}: retornamos un \lstinline{tyTree_MFA T} donde \lstinline{T} es
    el tipo telescopico constante.
  \end{itemize}
% TODO: En el implica es complicado porque usa a contains_u. No sé bien que hace
% esta función. Parece que al abstraer se utiliza un U : ArgsOf m. Si falla la
% abstracción devolvemos true. Creo que es un estilo de truco que se le ocurrió
% a Janno.
% X → Y
\item \lstinline{tyTree_imp X Y}: este es el caso más complicado porque requiere utilizar
  \lstinline{lift_in}. Maneja implicaciones sin dependecia entre tipos, mientras
  existe un caso más general para las dependencias.
  % Por qué el caso más general es más fácil? Tecnicamente debería ser más
  % difícil.
% forall x : X, F
\item \lstinline{tyTree_FAVal X F} donde \lstinline{X : Type}, \lstinline{X} puede seguir siendo el
  tipo original o haber sido reemplazado por un caso anterior, dado que el tipo
  se introduce antes que los valores de este mismo. En ambos casos operamos de
  la misma forma, retornando el tipo \lstinline{tyTree_FATeleType m F'} donde \lstinline{F'}
  es el lifteo de \lstinline{F}.
% forall A : Type, F
\item \lstinline{tyTree_FAType F}
  \begin{itemize}
  \item Si \lstinline{A} se encuentra bajo la mónada, es decir, hay una mención de
    \lstinline{A} que se encuentra dentro de \lstinline{M}, lo reemplazaremos por un
    \lstinline{A : MTele_Ty m} para algún \lstinline{m : MTele} cualquiera.
  \item Si no, \lstinline{A} seguira siendo un \lstinline{Type}.
  \end{itemize}
\end{enumerate}

Otro aspecto de esto es que este tipo de análisis por casos solo puede
realizarse en Mtac2 gracias a \lstinline{mmatch}, el \lstinline{match} monádico. Esto es
debido a la capacidad de analizar sintacticamente a los tipos.

Para empezar, notamos que los tipos \lstinline{A} y \lstinline{B} eran tipos primitivos de
Coq, y se convirtieron en \lstinline{MTele_Ty}, pero ambos tipos aparecían bajo la
mónada en la función.