\section{Coq}

En este capitulo introduciremos las características más relevantes del asistente de pruebas Coq. El objetivo de este capitulo es introducir todos los conceptos que utilizaremos más adelante, pero esto significa que no es una introducción completa.

\subsection{La historia de Coq}

El desarrollo de Coq comenzó en 1984 con el apoyo de INRIA como el trabajo de Gérard Huet y Thierry Coquand. En ese momento Coquand estaba implementado un lenguaje llamado \textit{Calculus of Constructions} cuando el 1991 una nueva implementación basada en un Calculus of Inductive Constructions extendido comenzó a ser desarrollado tomando el nombre de Coq.

Ahora mismo Coq es desarrollado por más de 40 desarrolladores activos y es reconocido como unos de los asistentes de prueba principales.

\subsection{Los lenguajes Coq}

Coq no es técnicamente un lenguaje de programación, si no un asistente de pruebas. Pero podemos encontrar múltiples lenguajes dentro de Coq que nos permiten expresarnos. 
\begin{itemize}
    % TODO: cita Gallina?
    \item \textit{Gallina}: este es el lenguaje de especificación de Coq. Permite desarrollar teorías matemáticas y probar especificaciones de programas. Utilizaremos extensivamente un lenguaje muy similar a este para definir nuestros programas en Mtac2.
    % TODO: cita Ltac?
    % TODO: donde hablo de tácticas?
    \item \textit{Ltac}: este el lenguaje en que se definen las \textit{tácticas} de Coq. Dado que Coq está centrado en las tácticas, Ltac es una de las partes centrales del aparato.
    \item \textit{Vernacular}.
\end{itemize}

Aunque Coq no es un lenguaje de programación propiamente dicho, este puede ser utilizado como un lenguaje de programación funcional. Estos programas serán especificados en Gallina. Dada la naturaleza de Coq como provador de teoremas, estos programas son funciones puras, es decir, no producen efectos secundarios y siempre terminan.

\subsection{La teoría de Coq}

\textit{Calculus of Inductive Constructions} es la base de Coq. Este es un cálculo lambda tipado de alto orden y puede ser interpretado como una extensión de la correspendencia Curry-Howard.

Llamaremos \textit{Terms} (o términos) a los elementos básicos de esta teoría. Terms incluye \textit{Type}, \textit{Prop}, variables, tuplas, funciones, entre otros. Estás son algúnas de las herramientas que utilizaremos para escribir nuestros programas.

% TODO: Set vs. Prop vs. Type??? Creo que no
% \subsection{Set vs. Prop vs. Type}

\subsection{Tipos de datos y Funciones}

Ahora aprenderemos a codificar nuestros programas funcionales en Coq. Lo primero que debemos entender es que operamos sobre \textit{términos} y algo es un término si tiene tipo. Coq provee muchos tipos predefinidos, por ejemplo \lstinline{unit}, \lstinline{bool}, \lstinline{nat}, \lstinline{list}, entre otros. A continuación estudiaremos cómo definir tipos y funciones.

Veamos cómo se define el tipo \lstinline{bool}:
\begin{lstlisting}
Inductive bool : Set :=
  | true : bool
  | false : bool.
\end{lstlisting}
Como podemos observar, es un tipo inductivo, especificado por la keyword \lstinline{Inductive}, con dos constructores \lstinline{true} y \lstinline{false}. De por sí, el tipo \lstinline{bool} no posee un significado hasta que nosotros lo proveamos de uno. Podemos ahora intentar definir algunos operadores booleanos.
\begin{lstlisting}
Definition andb (b1 b2:bool) : bool := if b1 then b2 else false.
Definition orb (b1 b2:bool) : bool := if b1 then true else b2.
Definition implb (b1 b2:bool) : bool := if b1 then b2 else true.
Definition negb (b:bool) := if b then false else true.
\end{lstlisting}
Las definiciones de funciones no recursivas comienzan con el keyword \lstinline{Definition}. La primera se llama \lstinline{andb} y toma dos booleanos como argumentos y retorna un booleano. Se utiliza la notación \lstinline{if b then x else y} para matchear sobre los booleanos de manera más fácil. Finalmente podemos definir una función más interesante.
\begin{lstlisting}
Definition Is_true (b:bool) :=
  match b with
    | true => True
    | false => False
  end.
\end{lstlisting}

Ahora, veamos un tipo con un ingrediente un poco más complicado, \lstinline{nat}.
\begin{lstlisting}
Inductive nat : Set :=
  | O : nat
  | S : nat -> nat.
\end{lstlisting}
Notemos que el constructor \lstinline{S} es una función que recibe un término de tipo {nat}, es decir, \lstinline{nat} es un tipo recursivo. Por ejemplo el término \lstinline{S (S O)} es de tipo \lstinline{nat} y representa al número 2.

Para continuar con este desarrollo, veamos el tipo de \lstinline{list} que es polimórfico.
\begin{lstlisting}
Inductive list (A : Type) : Type :=
  | nil : list A
  | cons : A -> list A -> list .
\end{lstlisting}
Este tipo es un tipo polimórfico dado que requiere de un \lstinline{A : Type}. Por ejemplo, una posible lista es \lstinline{cons (S O) nil : list nat} que representa a la lista con un único elemento 1 de tipo \lstinline{nat}.

Definiremos una función que añade un elemento a una lista.
\begin{lstlisting}
Definition add_list {A} (x : A) (l : list A) : list A :=
  cons x l.
\end{lstlisting}
Dado que el tipo \lstinline{A} puede ser inferido facilmente por Coq, utilizamos llaves a su alrededor para expresar que sea un argumento implícito. En el cuerpo de la función solo utilizamos \lstinline{cons}, uno de los constructores de \lstinline{list}, para añadir un elemento delante de \lstinline{l}.

Ahora nos interesa definir la función \lstinline{length} que retorna el largo de una lista.
\begin{lstlisting}
Fixpoint len {A} (l : list A) : nat :=
match l with
| [] => O
| x :: xs => S (len xs)
end.
\end{lstlisting}
Coq está diseñado de forma que necesitamos utilizar el keyword \lstinline{Fixpoint} para poder definir funciones recursiva. Aquí Coq está encontrando el argumento decreciente de la función \lstinline{len} y por eso acepta nuestra definición. El cuerpo de \lstinline{len} inspecciona a \lstinline{l} y lo \textit{matchea} con el caso correspondiente. Utilizamos \lstinline{S} y \lstinline{O}, los constructores de \lstinline{nat} para expresar el valor de retorno.

\subsection{Tipos dependientes}

Una de las herramientas más importantes que hay en Coq son los tipos dependientes. Estos nos permiten hablar de elementos que dependen de otros anteriores. Por ejemplo, puede ser de nuestro interés hablar de números positivos, en otras palabras, \lstinline{x : nat} tal que \lstinline{x <> O}. En este caso, \lstinline{x <> O} es una prueba que depende de \lstinline{x} y solo existirá cuando \lstinline{x} sea mayor a 0.

Para hablar de un ejemplo práctico de tipos dependientes, hemos elegido la función \lstinline{head} que retorna la cabeza de una lista. Comencemos con la versión más simple, donde utilizamos un valor default \lstinline{d} para el caso en que la lista es vacía.
\begin{lstlisting}
Definition head_d {A} (l : list A) (d : A) : A :=
  match l with
  | [] => d
  | x :: xs => x
  end.
\end{lstlisting}
El problema de esta solución es que a excepción de que \lstinline{d} sea un valor único, no hay manera de saber si la función retornó realmente la cabeza de la lista.

La segunda opción es utilizar el tipo \lstinline{option}.
\begin{lstlisting}
Inductive option A : Type :=
| None : option A
| Some : A -> option A.
\end{lstlisting}
Con este tipo auxiliar podemos reescribir \lstinline{hd}.
\begin{lstlisting}
Definition head_o {A} (l : list A) : option A :=
  match l with
  | [] => None
  | x :: xs => Some x
  end.
\end{lstlisting}
Esta solución es mejor que la anterior pero sigue sufriendo de una deficiencia. Dado que \lstinline{head_o} retorna un \lstinline{option} no sabemos si el resultado de esta función será realmente un elemento o si será el constructor vacío, por lo que todas las funciones que utilicen a \lstinline{head_o} deben también utilizar \lstinline{option}.
% TODO: citar? failure is not an option

Esto nos lleva a nuestra última solución. Esta requiere que nos aseguremos que la lista \lstinline{l} no es vacia, es decir, \lstinline{l <> []}. Pero para entenderla debemos ver dos cosas más: $\Sigma$\textit{-types} y \lstinline{Program}.
% TODO: sigma types.
Intuitivamente, los $\Sigma$\textit{-types} son tuplas donde el argumento de la derecha es dependiente del de la izquierda. A continuación, la definición.
\begin{lstlisting}
Inductive sig (A : Type) (P : A -> Prop) : Type :=
  exist : \forall x : A, P x -> {x : A | P x}
\end{lstlisting}
Se utiliza la notación \lstinline|{x : A \| P x}| para expresar \lstinline{sig A (fun x => P)}.
% TODO: está bien la notación?

\lstinline{Program} es una libreria que permite progamar en Coq como si fuera un lenguaje de programación funcional mientras que se utiliza una especificación tan rica como se desee y probando que el codigo cumple la especificación utilizando todo el mecanizmo de Coq. En nuestro caso utilizaremos \lstinline{Program} para codificar \lstinline{head} de una manera casi transparente.
\begin{lstlisting}
Program Definition head {A}
(l : list A | [] <> l ) : A :=
  match l with
  | [] => !
  | x :: xs => x
  end.
\end{lstlisting}
Como podemos observar, la única diferencia es que la signatura de \lstinline{head} especificamos que \lstinline{l} es una lista junto con una prueba que muestra que no es vacía.

% Este trabajo se centra en modificar meta-programas añadiendo tipos dependientes donde sea necesario, por eso es sumamente importante ejercitar este concepto.

% De esta manera, con la correspendencia Curry-Howard, un programa funcional puede ser una prueba de un teorema. Esta correspondecia está 
% gracias a Gallina podemos utilizarlo como uno. Gallina está fuertemente inspirado en OCaml, el lenguaje base de Coq y nuestros meta-programas estarán codificados en este. Esta programación en Coq es 

% TODO: mencionar metaprogramación, o sea Ltac2 pq no? o Meta-coq etc. Lo planeamos hacer en la parte de Mtac2, después de hablar de Mtac2.

\subsection{Tácticas}
% TODO: Cómo escribir tácticas de Coq, o sea Ltac. De esta forma se pueden comparar las metatacticas de Mtac2 con una referencia previa.