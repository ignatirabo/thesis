Mtac2 nos permite definir funciones monádicas. Estas cuentan con ciertas ventajas. Un ejemplo de un meta-programa es el siguiente.

\begin{coqe}
Definition list_max_nat :=
  mfix f (l: list nat) : l <> nil -> M nat :=
    mtmmatch l as l' return l' <> nil -> M nat with
    | [? e] [e] =m> fun nonE => M.ret e
    | [? e1 e2 l'] (e1 :: e2 :: l') =m> fun nonE =>
      let m := Nat.max e1 e2 in (* let pq nada de esto es monadico *)
      f (m :: l') cons_not_nil
    | [? l' r'] l' ++ r' => (* cualquier cosa *) 
    end.
\end{coqe}

Esta funcion calcula el maximo de una lista de numeros de $\natural$.
