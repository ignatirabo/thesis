\section{\lift}

Denominamos \lstinline{lift} a la función desarrollada en este trabajo.
La misma tiene la tarea de agregar dependencias a meta-programas de manera quasi automática: solo requiere un telescopio.

Lift se basa en analizar los tipos de las funciones y modificarlos añadiendo dependencias triviales en los tipos que se encuentran bajo la mónada, generando así nuevos meta-programas más generales.
El signficado real de este funcionamiento será trabajado en este capítulo a través de ejemplos que mostrarán el comportamiento de lift.

\subsection{Lifteando funciones simples}

Una de las funciones monádica más simples es \lstinline{ret : \forall A, A -> M A}, uno de los operadores monádicos.
Supongamos que nos interesa tener
\begin{lstlisting}
ret^ : \forall (A : nat -> nat -> Type), (\forall n n', A n n') -> (\forall n n', M (A n n'))
\end{lstlisting}

Ahora, nuestro trabajo es poder definir un telescopio que se amolde a la información que necesitamos agregar.
En este caso el telescopio es el siguiente: \lstinline{t := [n : nat ;> n' : nat]$_t$} dado que queremos que \lstinline{}{A} dependa de dos $\nat$. Efectivamente \lstinline{ret^ := lift ret t}.

Este ejemplo es simple porque no involucra funciones. Pero en ese caso, estudiemos que sucede con \lstinline{bind}.
\begin{lstlisting}
bind : \forall A B, M A -> (A -> M B) -> M B
\end{lstlisting}

Bind es el operador monádico más importante y es central en estos problemas.
Supongamos que queremos utilizamos el telescopio \lstinline{t := [T : Type ;> l : list T]$_t$} donde \lstinline{list T} es el tipo de las listas con elementos de tipo $T$.
Al momento de liftearlo, no es obvio cual debería ser el resultado, así que veamoslo.

\begin{lstlisting}
bind^ : \forall A B : \forall T : Type, list T -> Type,
         (\forall (T : Type) (l : list T), M (A T l)) ->
         (\forall (T : Type) (l : list T), A T l -> M (B T l)) ->
         (\forall (T : Type) (l : list T), M (B T l))
\end{lstlisting}

Lo más imporante es que si observamos la definición de esta función es sumamente simple.

\begin{lstlisting}
fun (A B : \forall T : Type, list T -> Type)
    (ma : \forall (T : Type) (l : list T), M (A T l))
    (f : \forall (T : Type) (l : list a), A T l -> M (B T l))
    (T : Type) (l : list T) =>
    bind (A T l) (B T l) (ma T l) (f T l)
\end{lstlisting}

Esta definición es efectivamente la que buscabamos y funciona perfectamente.
El otro aspecto que seguiremos observando es que Lift no genera información innecesaria en la función destino.