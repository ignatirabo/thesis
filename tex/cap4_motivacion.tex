Mtac2 nos permite definir funciones mónadicas. Estas cuentan con ciertas ventajas. Un ejemplo de un meta-programa es el siguiente.

\begin{coqe}
Definition list_max_nat :=
  mfix f (l: list nat) : l <> nil -> M nat :=
    mtmmatch l as l' return l' <> nil -> M nat with
    | [? e] [e] =m> fun nonE => M.ret e
    | [? e1 e2 l'] (e1 :: e2 :: l') =m> fun nonE =>
      let m := Nat.max e1 e2 in (* let pq nada de esto es monádco *)
      f (m :: l') cons_not_nil
    | [? l' r'] l' ++ r' => (* cualquier cosa *) 
    end.
\end{coqe}

Esta función calcula el máximo de una lista de números de \nat. Dado que en el último caso del \coqi{match} monádico analiza una expresión
con una función, y no un constructor, es imposible implementar esto sin Mtac2.

Ahora supongamos que deseamos parametrizar \nat y tener una función que acepte conjuntos. Sea
\begin{coqe}
Definition max (S: Set) : M (S -> S -> S) :=
  mmatch S in Set as S' return M (S' -> S' -> S') with
  | nat => M.ret Nat.max
  end.
\end{coqe}

la función que retorna la relación máximo en un conjunto $S$. A primera vista nuestra idea podría fucionar.

\begin{coqe}
Definition list_max (S: Set)  :=
  max <- max S; (* error! *)
  mfix f (l: list S) : l' <> nil -> M S :=
    mtmmatch l as l' return l' <> nil -> M S with
    | [? e] [e] =m> fun nonE=>M.ret e
    | [? e1 e2 l'] (e1 :: e2 :: l') =m> fun nonE =>
      m <- max e1 e2;
      f (m :: l') cons_not_nil
    end.
\end{coqe}

Al intentar interpreatar esto veremos que coqe no lo acepta. Esto es debido a la signatura de \coqi{bind}. Nuestro \coqi{mfix} no puede
unificarse a \coqi{M B}, ya que tiene tipo \coqi{f : forall (l: list S) : l' <> nil -> M S }.

\begin{coqe}
bind : forall A B, M A -> (A -> M B) -> M B.
\end{coqe}

Solucionar este problema específico no es un problema. Podemos codificar un nuevo bind que solucione el problema con un poco de esfuerzo.
El problema será que este solo soluciona el problema actual, y ante cualquier variación necesitaremos crear otro \coqi{bind} más.

Es por eso que nuestro proyecto es la codificación de un nuevo meta-programa \coqe{lift} que automaticamente puede generalizar meta-programas
con las dependencias necesarias para que sea utilizado en cualquier contexto.