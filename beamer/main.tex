\documentclass{beamer}

\usepackage[utf8]{inputenc}

\usetheme{Madrid}

% Comentario de Beta:
% Beta Ziliani: Es la motivación, esencialmente
% Beta Ziliani: O sea: Coq, necesitamos meta programas, Mtac, ejemplo en motivación
% Beta Ziliani: Idea general y vaga de la solución
% Beta Ziliani: Fin


\usepackage{xcolor}

\newcommand{\lift}{{\scshape Lift}}

\definecolor{ao}{rgb}{0.0, 0.5, 0.0}
\definecolor{applegreen}{rgb}{0.55, 0.71, 0.0}
\definecolor{asparagus}{rgb}{0.53, 0.66, 0.42}
\definecolor{auburn}{rgb}{0.43, 0.21, 0.1}
\definecolor{antiquefuchsia}{rgb}{0.57, 0.36, 0.51}
\definecolor{airforceblue}{rgb}{0.36, 0.54, 0.66}
\definecolor{cerulean}{rgb}{0.0, 0.48, 0.65}
\definecolor{brightmaroon}{rgb}{0.76, 0.13, 0.28}

\usepackage{listings,lstlangcoq,bold-extra}
\lstset{basicstyle=\ttfamily,language=Coq,showstringspaces=false}

%Information to be included in the title page:
\title[Generalización de Meta-programas]{Generalización de Meta-programas con Tipado Dependiente en Mtac2}
\author{Ignacio Tiraboschi}
\institute{UNC - FAMAF}
\date{2020}

\begin{document}

\frame{\titlepage}

\begin{frame}
\frametitle{Tabla de contenidos}
\tableofcontents
\end{frame}

\begin{frame}
\section{Coq}
\frametitle{Coq}
\framesubtitle{Asistentes de Prueba}

Los asistentes de prueba nos asisten en la demostración de teoremas de múltiples maneras.
\vspace{\baselineskip}

Existes muchos de ellos:
% TODO: Poner imágenes de diferentes asistentes de pruebas.
\vspace{\baselineskip}

Nosotros utilizaremos Coq.

\end{frame}

\begin{frame}
\frametitle{Coq}
% Coq es un asistente de prueba francés, desarrollado inicialmente por Thierry Coquand
Coq es uno de los asistentes de prueba más utilizados.
Podemos especificar \emph{Teoremas} y \emph{probarlos}.
Estas pruebas se centran en la modificación de metas a través de tácticas.
\vspace{\baselineskip}

Las tácticas se escriben en el metalenguaje Ltac y luego se concatenan formando una prueba.
% Ltac es el metalenguaje por defecto de Coq
\end{frame}

\begin{frame}[fragile]
\frametitle{Coq}

Las tácticas modifican metas o \emph{goals} y, pueden generar más metas.
\vskip \baselineskip

Tratemos probar \lstinline{Definition le_S (n : nat) : n <= S n}.
% Ejemplo 1. Utiliza induction esto genera dos metas nuevas:
% caso base y paso inductivo.
\vspace{\baselineskip}

\pause
Prueba:
\begin{lstlisting}
Proof.
induction n.
- apply (le_0_n 1).
- apply (le_n_S (n) (S n)). exact IHn.
Qed.
\end{lstlisting}

\end{frame}

\begin{frame}[fragile]
\frametitle{Coq}

\centering
Ahora veamos como definir programas en Coq.
\vskip \baselineskip

¿Cómo podemos definir listas?
\pause
\begin{lstlisting}
Inductive myList (A : Type) : Type :=
| myNil : myList A
| myCons : A -> myList A -> myList A.
\end{lstlisting}

\pause
¿Cómo definimos \lstinline{head} de manera correcta?

\pause
% Debemos utilizar un valor por defecto
\begin{lstlisting}
Definition head_d {A} (l : myList A) (d : A) : A :=
  match l with
  | myNil _ => d
  | myCons _ x xs => x
  end.
\end{lstlisting}

\pause
No siempre deseamos usar un valor por defecto.
% En el caso de los números naturales no existe un valor por defecto que podamos utilizar sin perder información.

\end{frame}

\begin{frame}[fragile]
\frametitle{Coq}

La otra manera de codificar esta función es utilizando \lstinline{option}.

\begin{lstlisting}
Definition head_o {A} (l : myList A) : option A :=
  match l with
  | myNil _ => None
  | myCons _ x xs => Some x
  end.
\end{lstlisting}

Esto tiene un problema: utilizando \lstinline{option} en \lstinline{head}, estamos obligados a continuar utilizandolo en el resto de las funciones.

\end{frame}

\begin{frame}[fragile]
\frametitle{Coq}

La ultima solución es mejor, pero recurre a la librería \lstinline{Program}.

\begin{lstlisting}
Program Definition head {A}
(l : myList A | myNil _ <> l ) : A :=
  match l with
  | myNil _ => !
  | myCons _ x xs => x
  end.
\end{lstlisting}

Esta solución hace uso de tipos dependientes!
% Se utilizan Sigma-types para añadir información que depende la lista l
% Esta información nos dice que l no es la lista vacia
\end{frame}

\begin{frame}
\section{Mtac2}
\frametitle{Mtac2}

El poder de Coq reside en su capacidad de taclear problemas grandes.
Esto se debe a que el uso de tácticas puede automatizarse y simplificarse.
\vspace{\baselineskip}

Sin embargo, las tácticas de Ltac no son tipadas.
Estas tienen muchas deficiencias: pueden ser ejecutadas en cualquier contexto y, no son fáciles de debuggear.
\end{frame}

\begin{frame}
\frametitle{Mtac2}

Mtac2 es uno de los metalenguajes de programación disponibles en Coq.
Su gran diferencia con Ltac son las tácticas tipadas.
\vspace{\baselineskip}

Las tacticas tipadas tienen una signatura y solo pueden ser utilizadas en el contexto correcto.
\vspace{\baselineskip}

También son más fáciles de debuggear!
% Ejemplo tácticas tipadas? Basicamente ver que podemos llamar cualquier táctica. Solo las tácticas correctas van a funcionar, las otras van a fallar sin decirnos por qué, solo que fallaron.
% En Mtac2 las tácticas solo son llamadas si encajan bien y, en el caso de fallar, tendremos un error informativo.
\end{frame}

\begin{frame}
\frametitle{Mtac2}

Definir funciones en Mtac2 es similar a Gallina.
La única diferencia es el uso de mónadas y, consiguientemente, el uso de operadores monádicos:
\begin{itemize}
  \item \lstinline{bind}: nos permite vincular o \emph{bindear} cómputos.
  \item \lstinline{ret}: utilizado para introducir un término a la mónada.
\end{itemize}
\vspace{\baselineskip}

Al comenzar a desarrollar metaprogramas más complicados, las signaturas se vuelven complicadas y el uso de estos operadores puede resultar en un problema.
\vskip \baselineskip

Antes de estudiar el problema, veamos como escribir metaprogramas.
\end{frame}

\begin{frame}[fragile]
\frametitle{Mtac2}
Utilizando \lstinline{bind} y \lstinline{ret} podemos armar nuestros cómputos.

\lstinline{bind : \forall A B : Type, M A -> (A -> M B) -> M B}
\lstinline{ret : \forall A : Type, A -> M A}
\vskip \baselineskip

También utilizaremos versiones monádicas de \lstinline{match} y del Fixpoint.
\vskip \baselineskip

Veamos el siguiente ejemplo.
\end{frame}

\begin{frame}[fragile]
\frametitle{Mtac2}

Este metaprograma evalúa números naturales.
Si están expresados como una suma podemos quitar las sumas con cero.

\begin{lstlisting}
Definition arith_eval : nat -> M nat :=
  mfix1 f (n : nat) : M nat :=
    mmatch n with
    | [? y] add O y =>
      y <- f y;
      ret y
    | [? x] add x O =>
      x <- f x;
      ret x
    | [? x y] add x y =>
      x <- f x;
      y <- f y;
      ret (add x y)
    | _ => ret n
  end.
\end{lstlisting}
\end{frame}

\begin{frame}
\frametitle{Mtac2}

Gracias a \lstinline{mmatch} podemos matchear en expresiones que no son constructores del tipo.
% En Coq solo podemos matchear en los constructores del tipo.
\vskip \baselineskip

Utilizamos \lstinline{mfix1} para expresar funciones recursivas en un argumento.
% También tenemos otras variantes: mfix2 y mfix3.
\vskip \baselineskip

Para nuestra motivación utilizaremos dos variantes:
\begin{itemize}
\item \lstinline{mtmmatch}
% Nos permite tener cuantificadores en la signatura retornada
\item \lstinline{mfix}
% Variante general de mfix_n
\end{itemize}
\end{frame}

\begin{frame}[fragile]
\frametitle{Mtac2}
\framesubtitle{Telescopios}

Lo último que necesitamos son telescopios.
\begin{lstlisting}
Inductive MTele : Type :=
| mBase : MTele
| mTele {X : Type} (F : X -> MTele) : MTele.
\end{lstlisting}

Esta estructura de dato nos permitirá expresar una secuencia de tipos o valores, posiblemente dependientes, y de largo arbitrario.
% El tipo \lstinline{MTele} crea una cadena de abstracciones.
% Este se codifica a través de funciones, lo que permite que sean dependientes, es decir, un telescopio puede tener elementos que dependan de elementos anteriores.
\vskip \baselineskip

Utilizaremos notación para referirnos a elementos de este tipo:
\begin{itemize}
  \item \lstinline{mBase = [$\;$]$_t$}
  \item \lstinline{[T$_0$ : Type ;> T$_1$ : T$_0$ -> Type ;> ...]$_t$}
\end{itemize}
\end{frame}

\begin{frame}[fragile]
\frametitle{Mtac2}
\framesubtitle{Telescopios}

Específicamente, estamos interesados en las funciones de telescopios.
Estas permiten observar un estilo de jerarquía.
\vskip \baselineskip

Necesitaremos la estructura de datos \lstinline{Sort}:
\begin{lstlisting}
Inductive Sort : Type := \Prop_sort | \Type_sort.
\end{lstlisting}
% Esta estructura es un reflejo de Prop y Sort, la utilizaremos como una abstracción
\vskip \baselineskip

También utilizaremos la función \lstinline{MTele_Sort}. Esta función derivará el tipo correspondiente al telescopio.
\end{frame}

\begin{frame}[fragile]
\frametitle{Mtac2}
\framesubtitle{Telescopios}

Supongamos que tenemos un telescopio \lstinline{m}:
\begin{lstlisting}
Definition m := [T : Type ;> l : list T ;> p : length l = 0]$_t$
\end{lstlisting}
\vskip \baselineskip

La expresión \lstinline{MTele_Sort \Type_sort m} se evalúa al tipo correspondiente:
\begin{lstlisting}
\forall (T : Type) (l : list T), length l = 0 -> Type : Type
\end{lstlisting}
% Retorna Type porque usamos Type_sort.
\lstinline{MTele_Sort} retorna un tipo!
\vskip \baselineskip

\pause
\centering
¿Qué término es una expresión del tipo?
\pause
\begin{lstlisting}
Definition Tm : MTele_Ty m := fun T l p => l = nil.
\end{lstlisting}
% l = nil podría ser cualquier tipo, no es necesario que dependa de los argumentos.
% Lo importante es que Coq acepta la definición de \lstinline{Tm}, lo que implica que esta definición efectivamente tiene el tipo esperado.
\end{frame}

\begin{frame}[fragile]
\frametitle{Mtac2}
\framesubtitle{Telescopios}

El elemento \lstinline{Tm} también es un tipo!
\vskip \baselineskip

Podemos definir el siguiente teorema y probarlo.

\begin{lstlisting}
Definition Tm_ex : forall T (l : list T) (p : (length l) = 0),
  l = nil.
Proof.
intros T l p. by apply length_zero_iff_nil.
Qed.
\end{lstlisting}

Entonces, la prueba \lstinline{Tm_ex} es un valor de \lstinline{Tm}.
Utilizando la función \lstinline{MTele_val} podemos especificarlo en Coq.
\begin{lstlisting}
Definition vm : MTele_val Tm := Tm_ex.
\end{lstlisting}

\end{frame}

\begin{frame}[fragile]
\section{Motivación}
\frametitle{Motivación}

Tenemos la siguiente función que calcula el máximo de una lista de números naturales.
\begin{lstlisting}
Definition list_max_nat :=
  mfix f (l: list nat) : l <> nil -> M nat :=
    mtmmatch l as l' return l' <> nil -> M nat with
    | [? e] [e] =m> fun P => M.ret e
    | [? e1 e2 l'] (e1 :: e2 :: l') =m> fun P =>
      let x := Nat.max e1 e2 in
      f (x :: l') _
    | [? l' r'] app l' r' =m> fun P =>
      match app_not_nil l' r' P with
      | inl P' => f l' P'
      | inr P' => f r' P'
      end
    end.
\end{lstlisting}

\end{frame}

\begin{frame}[fragile]
\frametitle{Motivación}

Supongamos que deseamos una función general, que funciones con múltiples tipos de listas.
\vskip \baselineskip

Sea \lstinline{max} la función que retorna la función que calcula el máximo entre dos elementos de un cierto tipo \lstinline{S}.
\begin{lstlisting}
Definition max (S: Set) : M (S -> S -> S) :=
  mmatch S in Set as S' return M (S' -> S' -> S') with
  | nat => ret Nat.max
  | bool => ret (fun b b' => if b then b else b')
  | _ => ret (fun x _ => x)
  end.
\end{lstlisting}

Ahora, podemos intentar integrar esta función a \lstinline{list_max_nat}.
\end{frame}

\begin{frame}[fragile]
\frametitle{Motivación}

Nueva versión:
\begin{lstlisting}
Definition list_max (S: Set) :=
  max <- max S; (* error! *)
  mfix f (l: list S) : l <> nil -> M S :=
    mtmmatch l as l' return l' <> nil -> M S with
    | [? e] [e] => fun P => M.ret e
    | [? e1 e2 l'] (e1 :: e2 :: l') => fun P =>
      m <- max e1 e2;
      f (m :: l') _
    | ...
    end.
\end{lstlisting}

Tenemos un problema de tipos!
\end{frame}

\begin{frame}[fragile]
\frametitle{Motivación}

La signatura de \lstinline{max S} es \lstinline{M (S -> S -> S)}, mientras que la del fixpoint es \lstinline{f : forall (l: list S),  l' <> nil -> M S}.
\vskip \baselineskip

Dado que \lstinline{bind : \forall (A B : Type), M A -> (A -> M B) -> M B} es imposible que vinculemos a \lstinline{max S} con el punto fijo.
\vskip \baselineskip

Debemos pensar una nueva forma de hacer esto.
La primera opción es generalizar a \lstinline{bind}.
En ese caso, debemos hacer que su retorno se asemeje al punto fijo. Es decir, que permita retornar una función.
\end{frame}

\begin{frame}
\section{Lift}
\frametitle{Lift}

Nosotros decidimos abordar una alternativa mucho más general.
Esta es la de la generalización casi automática de funciones.
Utilizando telescopios, podemos indicar las dependencias de interés y generalizar.
\vskip \baselineskip

Esta generalización se hace a través del análisis recursivo en el tipo de la función.
Para eso utilizaremos una estructura de datos adicional: \lstinline{TyTree}.

\end{frame}

\begin{frame}[fragile]
\frametitle{Motivación}

El tipo \lstinline{TyTree} es un reflejo de los tipos de Coq.
\begin{lstlisting}
Inductive TyTree : Type :=
| \tyTree_val$\!$ {m : MTele} (T : MTele_Ty m) : TyTree
| \tyTree_M$\!$ (T : Type) : TyTree
| \tyTree_MFA$\!$ {m : MTele} (T : MTele_Ty m) : TyTree
| \tyTree_In$\!$ (s : Sort) {m : MTele} (F : accessor m -> s) : TyTree
| \tyTree_imp$\!$ (T : TyTree) (R : TyTree) : TyTree
| \tyTree_FATeleVal$\!$ {m : MTele} (T : MTele_Ty m)
  (F : \forall t : MTele_val T, TyTree) : TyTree
| \tyTree_FATeleType$\!$ (m : MTele) (F : \forall (T : MTele_Ty m), TyTree) : TyTree
| \tyTree_FAVal$\!$ (T : Type) (F : T -> TyTree) : TyTree
| \tyTree_FAType$\!$ (F : Type -> TyTree) : TyTree
| \tyTree_base$\!$ (T : Type) : TyTree
.
\end{lstlisting}
\end{frame}

\begin{frame}[fragile]
\frametitle{Lift}

La función \lift\ toma una función \lstinline{f}, un elemento \lstinline{TyTree} que describe su tipo, y un telescopio \lstinline{m}.
Con estos elementos es capaz de agregar las dependencias de \lstinline{m} en \lstinline{f}.
\vskip \baselineskip

En el caso de la motivación, deseamos liftear \lstinline{bind} y podemos definir un telescopio adecuado:
\begin{lstlisting}
m : MTele := [l : list S ;> p : l <> nil]$_t$
\end{lstlisting}

Como vimos antes, el fixpoint de \lstinline{list_max} tiene signatura:
\begin{lstlisting}
f : forall (l: list S),  l' <> nil -> M S
\end{lstlisting}
\end{frame}

\begin{frame}[fragile]
\frametitle{Lift}

La signatura de la función \lstinline{bind} es:
\begin{lstlisting}
bind : \forall (A B : Type), M A -> (A -> M B) -> M B
\end{lstlisting}
Reescribiendo la signatura con \lstinline{TyTree} conseguimos:
\begin{lstlisting}
Definition bindTyTree :=
  tyTree_FAType (fun A : Type =>
    tyTree_FAType (fun B : Type =>
      tyTree_imp
        (tyTree_M A)
        (tyTree_imp
          (tyTree_imp (tyTree_base A) (tyTree_M B))
          (tyTree_M B)))).
\end{lstlisting}
\vskip \baselineskip

\pause
\begin{center}
Estamos listos para liftear \lstinline{bind}!
\end{center}
\end{frame}

\end{document}