\documentclass{beamer}

\usepackage[utf8]{inputenc}

% Comentario de Beta:
% Beta Ziliani: Es la motivación, esencialmente
% Beta Ziliani: O sea: Coq, necesitamos meta programas, Mtac, ejemplo en motivación
% Beta Ziliani: Idea general y vaga de la solución
% Beta Ziliani: Fin


\usepackage{xcolor}

\newcommand{\lift}{{\scshape Lift}}

\definecolor{ao}{rgb}{0.0, 0.5, 0.0}
\definecolor{applegreen}{rgb}{0.55, 0.71, 0.0}
\definecolor{asparagus}{rgb}{0.53, 0.66, 0.42}
\definecolor{auburn}{rgb}{0.43, 0.21, 0.1}
\definecolor{antiquefuchsia}{rgb}{0.57, 0.36, 0.51}
\definecolor{airforceblue}{rgb}{0.36, 0.54, 0.66}
\definecolor{cerulean}{rgb}{0.0, 0.48, 0.65}
\definecolor{brightmaroon}{rgb}{0.76, 0.13, 0.28}

\usepackage{listings,lstlangcoq,bold-extra}
\lstset{basicstyle=\ttfamily,language=Coq,showstringspaces=false}

%Information to be included in the title page:
\title{Generalización de Meta-programas con Tipado Dependiente en Mtac2}
\author{Ignacio Tiraboschi}
\institute{Universidad Nacional de Córdoba - FAMAF}
\date{2020}

\begin{document}

\frame{\titlepage}

\begin{frame}
\frametitle{Asistentes de Prueba}

Los asistentes de prueba nos asisten en la demostración de teoremas de múltiples maneras.
\vspace{\baselineskip}

Existes muchos de ellos: Coq, Isabelle, Agda, Lean, HOL4.
% TODO: Poner imágenes de diferentes asistentes de pruebas.
\vspace{\baselineskip}

Nosotros utilizaremos Coq.

\end{frame}

\begin{frame}
\frametitle{Coq}
% Coq es un asistente de prueba francés, desarrollado inicialmente por Thierry Coquand
Coq es uno de los asistentes de prueba más utilizados.
Podemos especificar \emph{teoremas} y \emph{probarlos}.
\vspace{\baselineskip}

Estas pruebas se centran en la modificación de metas a través de tácticas.
\vspace{\baselineskip}

Las tácticas se escriben en el metalenguaje Ltac y luego se concatenan formando una prueba.
% Ltac es el metalenguaje por defecto de Coq
% Veamos qué son las tácticas
\end{frame}

\begin{frame}[fragile]
\frametitle{Coq}

Las tácticas modifican metas (también llamadas \emph{goals}) y, pueden generar más metas.
\vskip \baselineskip

% Centremosnos en el siguiente teorema.
\lstinline{Definition le_S (n : nat) : n <= S n.}
% Este teoema nos dice que un número natural es menor a su sucesor.
% Ejemplo 1. Utiliza induction esto genera dos metas nuevas:
% caso base y paso inductivo.
\pause
\begin{lstlisting}
Proof.
induction n.
- apply (le_0_n 1).
- apply (le_n_S (n) (S n)). exact IHn.
Qed.
\end{lstlisting}
% Veamos esta prueba en vivo.

\end{frame}

\begin{frame}
\frametitle{Coq}

El poder de Coq reside en su capacidad de taclear problemas grandes.
Esto se debe a que el uso de tácticas puede automatizarse y simplificarse.
\vspace{\baselineskip}

\pause
Uno de los problemas de Ltac es la falta de tácticas tipadas.
Estas tienen muchas deficiencias: pueden ser utilizadas en cualquier contexto y, no son fáciles de debuggear.
% Veamos un ejemplo en vivo: ejemplo sintipar?
\end{frame}

\begin{frame}
\frametitle{Mtac2}

Mtac2 es uno de los metalenguajes de programación disponibles en Coq.
Su gran diferencia con Ltac son las tácticas tipadas.
% Otros disponibles son Ltac2, Rtac, Template-Coq y Coq-Elpi, aparte de Ltac obviamente.
% La gran diferencia entre estos meta-lenguajes con Mtac2 es que no tienen tácticas tipadas. Brindan menos garantías.
\vspace{\baselineskip}

\pause
Las tacticas tipadas tienen una signatura y solo pueden ser utilizadas en el contexto correcto.
\vspace{\baselineskip}

\pause
También son más fáciles de debuggear!
% Ejemplo tácticas tipadas? Basicamente ver que podemos llamar cualquier táctica. Solo las tácticas correctas van a funcionar, las otras van a fallar sin decirnos por qué, solo que fallaron.
% En Mtac2 las tácticas solo son llamadas si encajan bien y, en el caso de fallar, tendremos un error informativo.
% Podemos utilizar prints y excepciones.
\end{frame}

\begin{frame}
\frametitle{Mtac2}

Definir funciones en Mtac2 es similar a Gallina.
La única diferencia es el uso de mónadas y, consiguientemente, el uso de operadores monádicos:
\begin{itemize}
    \item \lstinline{bind}: nos permite vincular o \emph{bindear} cómputos.
    \item \lstinline{ret}: utilizado para introducir un término a la mónada.
\end{itemize}
\vspace{\baselineskip}

\pause
Al comenzar a desarrollar metaprogramas más complicados, las signaturas se vuelven complicadas y el uso de estos operadores puede resultar en un problema.
\vskip \baselineskip

Antes de estudiar el problema, estudiemos como escribir metaprogramas.
\end{frame}

\begin{frame}[fragile]
\frametitle{Mtac2}

\centering
¿Cómo podemos definir listas?
\vskip \baselineskip

\pause
\begin{lstlisting}
Inductive myList (A : Type) : Type :=
| myNil : myList A
| myCons : A -> myList A -> myList A.
\end{lstlisting}

\pause
¿Cómo definimos \lstinline{head} de manera correcta?

\pause
% Debemos utilizar un valor por defecto
\begin{lstlisting}
Definition head_d {A} (l : myList A) (d : A) : A :=
  match l with
  | myNil _ => d
  | myCons _ x xs => x
  end.
\end{lstlisting}

\pause
No siempre deseamos usar un valor por defecto.
% En el caso de los números naturales no existe un valor por defecto que podamos utilizar sin perder información.

\end{frame}

\begin{frame}[fragile]
\frametitle{Mtac2}

La otra manera de codificar esta función es utilizando \lstinline{option}.

\begin{lstlisting}
Definition head_o {A} (l : myList A) : option A :=
  match l with
  | myNil _ => None
  | myCons _ x xs => Some x
  end.
\end{lstlisting}

Esto tiene un problema: utilizando \lstinline{option} en \lstinline{head}, estamos obligados a continuar utilizandolo en el resto de las funciones.

\end{frame}

\begin{frame}[fragile]
\frametitle{Mtac2}

La ultima solución es mejor, pero recurre a la librería \lstinline{Program}.

\begin{lstlisting}
Program Definition head {A}
(l : myList A | myNil _ <> l ) : A :=
  match l with
  | myNil _ => !
  | myCons _ x xs => x
  end.
\end{lstlisting}

Esta solución hace uso de tipos dependientes!
% Se utilizan Sigma-types para añadir información que depende la lista l
% Esta información nos dice que l no es la lista vacia
\end{frame}

\begin{frame}
\frametitle{Motivación}

El problema que nos interesa es el siguiente.



\end{frame}

\begin{frame}
\frametitle{Motivación}
\end{frame}

\end{document}