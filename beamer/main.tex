\documentclass{beamer}

\usepackage[utf8]{inputenc}

\usepackage{xcolor}

\newcommand{\lift}{{\scshape Lift}}

\definecolor{ao}{rgb}{0.0, 0.5, 0.0}
\definecolor{applegreen}{rgb}{0.55, 0.71, 0.0}
\definecolor{asparagus}{rgb}{0.53, 0.66, 0.42}
\definecolor{auburn}{rgb}{0.43, 0.21, 0.1}
\definecolor{antiquefuchsia}{rgb}{0.57, 0.36, 0.51}
\definecolor{airforceblue}{rgb}{0.36, 0.54, 0.66}
\definecolor{cerulean}{rgb}{0.0, 0.48, 0.65}
\definecolor{brightmaroon}{rgb}{0.76, 0.13, 0.28}

\usepackage{listings,lstlangcoq,bold-extra}
\lstset{basicstyle=\ttfamily,language=Coq,showstringspaces=false}

%Information to be included in the title page:
\title{Generalización de Meta-programas con Tipado Dependiente en Mtac2}
\author{Ignacio Tiraboschi}
\institute{Universidad Nacional de Córdoba - FAMAF}
\date{2020}

\begin{document}

\frame{\titlepage}

\begin{frame}
\frametitle{Asistentes de Prueba}

Los asistentes de prueba nos asisten en la demostración de teoremas de múltiples maneras.
\vspace{\baselineskip}

Existes muchos de ellos:
% TODO: Poner imágenes de diferentes asistentes de pruebas.
\vspace{\baselineskip}

Nosotros utilizaremos Coq.

\end{frame}

\begin{frame}
\frametitle{Coq}
% Coq es un asistente de prueba francés, desarrollado inicialmente por Thierry Coquand
Coq es uno de los asistentes de prueba más utilizados.
Podemos especificar \emph{Teoremas} y \emph{probarlos}.
Estas pruebas se centran en la modificación de metas a través de tácticas.
\vspace{\baselineskip}

Las tácticas se escriben en el metalenguaje Ltac y luego se concatenan formando una prueba.
% Ltac es el metalenguaje por defecto de Coq
\end{frame}

\begin{frame}
\frametitle{Coq}

Las tácticas modifican metas o \emph{goals} y, pueden generar más metas.
\vspace{\baselineskip}

Tratemos de probar \lstinline{Definition le_S (n : nat) : n <= S n.}
% Ejemplo 1. Utiliza induction esto genera dos metas nuevas:
% caso base y paso inductivo.
\vspace{\baselineskip}

También podemos definir tipos de manera inductiva con el operador \lstinline{Inductive}.
\vspace{\baselineskip}

Así, podemos definir, por ejemplo, listas en Coq.
% Ejemplo definir listas

\end{frame}

\begin{frame}
\frametitle{Coq}

El poder de Coq reside en su capacidad de taclear problemas grandes.
\vspace{\baselineskip}



\end{frame}

\begin{frame}
\frametitle{Mtac2}

Mtac2 es uno de los metalenguajes de programación disponibles en Coq.

Su gran diferencia son las tácticas tipadas.

% Ejemplo tácticas tipadas?
\end{frame}

\begin{frame}
\frametitle{Mtac2}

Definir funciones en Mtac2 es similar a Gallina.
La única diferencia es el uso de mónadas.
\vspace{\baselineskip}

\end{frame}

\begin{frame}
\frametitle{Motivación}
\end{frame}

\end{document}